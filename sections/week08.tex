% Techniken aufzählen wie Web-UI sinnvoll für Farbenblindeheit optimiert werden können.
% erklären warum Optimierung für Farbblindheit und Optimierung des Farbkontrasts von Web-UI Elementen nicht das Gleiche ist .
% Überprüfungsvorschläge des Chrome Accessibility Checks in einem Kontext in Handlungsempfehlungen übersetzen.
% erklären warum Zoombarkeit von Web-Uis wichtig ist, und welche Praktiken vermieden werden sollten.
% erklären warum Animationen in Web-UI abgestellt werden können sollen.
% die Accessibility von Formularen und Tabellen beurteilen und optimieren.

\section{Accessibility}
\textbf{Stolze Regeln:} Bilder mit Alt-Text, Keine Informationen ausschließlich in Farbe dargestellt, Vorder- und Hintergrund bei reduzierter Farb- und Kontrastwahrnehmung in Standardansicht deutlich unterscheidbar, Skalierung der Schrift über Funktionen des Browsers möglich (arbeiten mit: em/rem), 

\subsection{Farbdesign}
Die Farben für Farbenblinde simulieren, um unerkennbare Unterscheidungen zu vermeiden. Mehrfach-Codierungen verwenden: Fabre und Form, etc. \textbf{Farbkontraste:} (Wichtig für 50+), Level AA=4.5:1, AAA=7:1.

\subsection{Zoombarkeit}
Emfpehlung: Nicht unberbinen. 

\subsection{Animationen}
Empfehlung: sollten abstellbar sein. Verhinderung der Auslösung von Epilepsie und Migräne.

\subsection{Medien}
Sollten immer einen Alternativen-Text haben.

\subsection{Screen Reader}
Keine Heading-Levels auslassen, Semantic Elements richtig nutzen, Skip-Links am Anfang der Seite.