% Die Bedeutung von Nielsens „1st Rule of Usability“ („don‘t Listen to Users“) erklären. Sie können zur Nützlichkeit von Fokus Gruppen beim User Research Auskunft geben.
% Den Unterschied von Problem-Space und Solution-Space erklären und warum dieser Unterschied bedeutet, dass Nutzer keine gute Quelle von Design-Vorschlägen sein sollten.
% Erklären warum bei Nutzerforschung neben der Toolverwendung auch Eigenschaften,
% Nutzer-Aufgaben und der Kontext dokumentiert werden sollten.
% Den Wert und die Elemente von Szenarios erklären.
% Techniken zur Bewertung und Optimierung der Navigationsunterstützung (Fragebogen, Card-Sort, Tree-Testing) in Web-Seites erklären und sinnvoll einsetzen.

\section{UX-Research, Information Architecture}











