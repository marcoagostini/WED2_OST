% Die Bedeutung von Nielsens „1st Rule of Usability“ („don‘t Listen to Users“) erklären. Sie können zur Nützlichkeit von Fokus Gruppen beim User Research Auskunft geben.
% Den Unterschied von Problem-Space und Solution-Space erklären und warum dieser Unterschied bedeutet, dass Nutzer keine gute Quelle von Design-Vorschlägen sein sollten.
% Erklären warum bei Nutzerforschung neben der Toolverwendung auch Eigenschaften, Nutzer-Aufgaben und der Kontext dokumentiert werden sollten.
% Den Wert und die Elemente von Szenarios erklären.
% Techniken zur Bewertung und Optimierung der Navigationsunterstützung (Fragebogen, Card-Sort, Tree-Testing) in Web-Seites erklären und sinnvoll einsetzen.

\section{UX-Research, Information Architecture}
\subsection{Usability Kriterien nach Nielsen}
\textbf{(1)} Sichtbarkeit des System-Status \textbf{(2)} Enger Bezug zwischen System und realer Welt \textbf{(3)} Nutzerkontrolle und Freiheit \textbf{(4)} Konsistenz \& Konformität mit Standards \textbf{(5)} Fehler-Vorbeugung \textbf{(6)} Besser Sichtbarkeit als Sich-erinnern-müssen \textbf{(7)} Flexibilität und Nutzungseffizienz \textbf{(8)} Ästhetik und minimalistischer Aufbau \textbf{(9)} Nutzern helfen, Fehler zu bemerken, zu diagnostizieren und zu beheben \textbf{(10)} Hilfe und Dokumentation\\
\textbf{\color{blue}Card-Sort:} Evaluation der Informationsarchitektur der Seite, Teilnehmer sortieren Themen in Gruppen.
\textbf{\color{blue}Tree-Testing:}  Evaluation wie schnell User Informationen finden, Suchaufgabe, welche die User ausführen.











