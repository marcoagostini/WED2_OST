% verstehen den unterschied zwischen Server und Client JavaScript Code.
% kennen den Unterschied zwischen klassischen und event-driven Web Services.
% können die Konzepte von Node.js anwenden.
% Callbacks / Events / NPM • können mit Node.js einen rudimentären Server implementieren.

\section{NodeJS}
Was muss ein Webserver können: HTTP Anfragen annehmen, Actions ausführen basierend auf URL, HTTP Antworten senden. \textbf{Eigenschaften:} JS runtime, event-driven non-blocking I/O model, NPM.