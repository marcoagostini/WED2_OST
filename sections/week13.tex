% die Aufgabe und den Einsatzbereich von (CSS) Präprozessoren erklären.
% Sass (SCSS) Code interpretieren (inkl. Variablen, Nesting, Mixing, Partials und Funktionen) und den aus Sass/SCSS Code den zugehörigen CSS Code generieren. 
% Fehler in Sass/SCSS Code identifizieren und beheben, bzw. Lücken im Code füllen.  
% mit Sass/SCSS Einheiten rechnen. 
% zwei spezifische Einsatzbereiche von PostCSS nennen. 
% erklären in welchen Kontexten es sinnvoll ist Web-Build-Tools einzusetzen und zwei spezifische Features dieser Build Tools   nennen
% die in der Vorlesung behandelten Präprozessoren, Testing Tools, Build-Tools etc.  der   korr ekten   Kategorie zuordnen.

\section{WebDevOps}
\subsection{SASS}
Ist ein Prä-Prozessor Compiler für CSS.
\textbf{Features:} Variablen, Verschachtelung, Partials, Mixins, Extends, Programmierung.

\begin{lstlisting}[style=htmlcssjs]
@use 'base'; // imports module/partial "_base.scss"
$font-stack: Helvetica, sans-serif;
$primary-color: #333;
nav {
  ul {
    margin: 0;
  }
}
@mixin theme($theme: DarkGray) { // definition mixin
  background: $theme;
  color: #fff;
}
.info { // use of mixin
  @include theme;
}  
%message-shared { // Parent (nesting)
  border: 1px solid #ccc;
  padding: 10px;
  color: #333;
}
.message { // child
  @extend %message-shared;
}
\end{lstlisting}