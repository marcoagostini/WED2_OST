% die folgenden Begriffe erklären: Flexible Layout, Responsive Layout, Graceful Degration,
% Progressive Enhancement, Mobile First Layout, CSS Resets / CSS Normalization. 
% CSS Media-Queries erstellen. 
% Flexible Layout mit dem Box-Model erstellen, interpretieren und korrigieren. inklusive calc(), min(), max(), clamp(); position absolute/relative 
% Flexbox Layouts erstellen, interpretieren und korrigieren. 
% CSS Grid Layouts erstellen, interpretieren und korrigieren.

\section{Responsive Layout}
\textbf{Flexibles Layout:} Dynamisches (grössenadaptives) Layout welches sich ohne Media-Queries umsetzen lassen. \\
\textbf{Reponsives Layout:} Dynamisches Layout welches für unterschiedliche Geräte, Bereiche von Display-Grössen und unterschiedliche Medien separates Layouts definiert.

\subsection{Media Queries}
Kann fast alle Einheiten verwenden (em=best Accessibility). \textbf{Typische Triggerpunkte:} 480px, 768px, 992px.

\begin{lstlisting}[style=htmlcssjs]
@media screen { [width|min-width|max-width]:700px) {}
@meida (min-width: 20em) and|not (max-width: 30em) {}
\end{lstlisting}

\subsection{ViewPort}
Mobile Geräte benötigen Deklaration des Viewports für Media Queries.

\subsection{CSS}
\subsubsection{calc()}
Kann verschiedene Werte direkt im CSS verrechnen (Ausser Media Queries).
\subsubsection{min(), max()}
Vergleich mehrere Werte und gibt den kleinsten/grössten zurück.
\subsubsection{clamp()}
min val, actual size, max val. 

\subsection{Floats}
Nicht mehr aktuell und nicht mehr zu verwenden.

\subsection{Custom Attributes}
\begin{lstlisting}[style=htmlcssjs]
--color-light-red: #ffeaea; // variable
--color: var(--color-dark-violet); // reference
\end{lstlisting}






